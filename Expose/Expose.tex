\documentclass[a4paper,11pt]{article}

\usepackage[ngerman]{babel}
\usepackage{graphicx}
\usepackage{eso-pic}
\usepackage{tabularx}
\usepackage{stackengine}
\usepackage{ragged2e}
\usepackage{url}
\usepackage{hyperref}
\usepackage{microtype}
\usepackage{unicode-math}
\setmathfont{Latin Modern Math}     % gegen Warnung wegen Arial Schriftart
\usepackage[left=3cm, right=5cm, top=1.5cm, bottom=2cm]{geometry}
\usepackage{setspace}
\onehalfspacing                 % Zeilenabstand
\usepackage{fontspec}
\setmainfont{Arial}


\author{}
\date{}
\title{{\fontsize{26pt}{30pt}\selectfont Seminarfach} \\[3cm] {\fontsize{30pt}{40pt}\selectfont Exposé}}


\newcommand{\StartBackground}{
\newgeometry{top=6.5cm,bottom=1.5cm,left=4cm,right=2cm}
\AddToShipoutPictureBG{
\put(0,0){
\parbox[b][\paperheight]{\paperwidth}{
\vfill
\centering
\includegraphics[width=\paperwidth,height=\paperheight]{../Background.png}
\vfill}}}}
\newcommand{\StopBackground}{\newpage\restoregeometry\ClearShipoutPictureBG}

\pagestyle{empty}
\pagenumbering{gobble}


% Zitation
\usepackage{csquotes}

\usepackage[
style=authoryear,
backend=biber,
natbib=true,
maxcitenames=1,
mincitenames=1,
maxbibnames=99,
minbibnames=99,
citetracker=context,
giveninits=true
]{biblatex}

\DeclareDelimFormat{finalnamedelim}{\space und\space}
\renewcommand*{\postnotedelim}{\addcolon\space}
\renewcommand*{\multicitedelim}{\addsemicolon\space}
\DeclareFieldFormat{postnote}{S.#1}
\DefineBibliographyStrings{ngerman}{andothers = {et\addabbrvspace al\adddot}}
\renewbibmacro*{name:andothers}{}
\DeclareFieldFormat[book,thesis]{title}{\mkbibemph{#1}}
\DeclareFieldFormat[article]{title}{\mkbibquote{#1}}
\DeclareNameAlias{author}{family-given}
\DeclareNameAlias{editor}{family-given}
\setlength{\bibhang}{1.25cm}
\setlength{\bibitemsep}{0.5\baselineskip}
\setlength{\bibparsep}{0pt}
\DefineBibliographyStrings{ngerman}{editor = {Hrsg\adddot}, editors = {Hrsg\adddot}}
\DeclareLabeldate{\field{date} \field{year}}
\ExecuteBibliographyOptions{uniquename=false}
\DefineBibliographyStrings{ngerman}{urlseen = {Zugriff am}}
\renewcommand*{\bibfont}{\fontsize{9pt}{11pt}\selectfont}      % Größe der Verzeichnisses
\newcommand{\ebd}[1][]{\textit{(ebd.\if\relax\detokenize{#1}\relax\else~S.#1\fi)}}  % ebd. Zitation (geht leider nicht automatisch)
\AtBeginBibliography{\interlinepenalty=10000}
\addbibresource{../Quellen/Quellen.bib}                 % --> Ändern wenn in Unterordnern


%---------------------------------------------------------------------------------
% Titelseite / Benotung

\begin{document}
\justifying                 % => für Blocksatz
\StartBackground

\maketitle
\vspace{4cm}
\begin{table}[h!]
    \renewcommand{\arraystretch}{3}
    \begin{tabularx}{\textwidth}{>{\raggedright\arraybackslash}p{5cm}
                                 >{\raggedright\arraybackslash}X}
        \textbf{Thema:}&Entstehungsmechanismen und strukturelle Merkmale von Zentralbergen in Impaktkratern, Vulkanen und Äquatorialkämmen im Sonnensystem\\
        \textbf{Verfasser:}&Julian Baden\\
        \textbf{Fachlehrkraft und Kursbezeichnung:}&
        \end{tabularx}
\end{table}


\newpage
\noindent\textbf{Gymnasium Mellendorf\\Fritz-Sennheiser-Platz 2\\30900 Wedemark}\\
Schuljahr:  2025/26\\

{\fontsize{26pt}{20pt}\selectfont\begin{center} Seminarfach \end{center}}

\noindent\rule{\linewidth}{0.5pt}

\begin{table}[h!]
    \renewcommand{\arraystretch}{2}
    \begin{tabularx}{\textwidth}{>{\raggedright\arraybackslash}p{4.5cm}
                                 >{\raggedright\arraybackslash}X}
        Verfasser:&Julian Baden\\
        Thema:&Entstehungsmechanismen und strukturelle Merkmale von Zentralbergen in Impaktkratern, Vulkanen und Äquatorialkämmen im Sonnensystem\\
        Fachlehrkraft und Kursbezeichnung:
        \end{tabularx}
\end{table}

\noindent\rule{\linewidth}{0.5pt}\\[0.5cm]
\textbf{Abgabetermin: 19.02.2026 zu Beginn der Seminarfachsitzung um 14 Uhr}\\[0.3cm]
\rule{\linewidth}{0.5pt}\\

\begin{tabular}{l  l}
    &\\
    \textbf{Benotung:}&\fbox{\rule[-1.5em]{0pt}{4em}\hspace{3cm}}\\
    &\\[0.3cm]
\end{tabular}

\hspace{3cm} Verfasser*in\\[1cm]

\hspace{1cm} \stackunder[4pt]{\rule{4.5cm}{0.6pt}}{\makebox[4.5cm]{Datum}}
\hspace{2cm} \stackunder[4pt]{\rule{5.5cm}{0.6pt}}{\makebox[5.5cm]{Unterschrift der Lehrkraft}}


%---------------------------------------------------------------------------------
% Eigenständigkeit --> HINTEN ANHÄNGEN!!!

\newpage
\noindent{\fontsize{14pt}{16pt}\selectfont \textbf{Erklärung des Verfassers}}\\[0.5cm]
Hiermit erkläre ich, dass ich die vorliegende Arbeit selbstständig angefertigt,
keine anderen als die angegebenen Hilfsmittel benutzt und die Stellen der Facharbeit,
die im Wortlaut oder im wesentlichen Inhalt aus anderen Werken oder dem Internet entnommen wurden,
mit genauer Quellenangabe kenntlich gemacht habe.\\[0.8cm]

\noindent Verfasser:\quad \\[0.8cm]

\hspace{1cm} \stackunder[4pt]{\rule{4.5cm}{0.6pt}}{\makebox[4.5cm]{Datum}}
\hspace{2cm} \stackunder[4pt]{\rule{5.5cm}{0.6pt}}{\makebox[5.5cm]{Unterschrift}}

\StopBackground


%---------------------------------------------------------------------------------
% Inhaltsverzeichnis


\normalsize

\pagestyle{plain}
\pagenumbering{Roman}
\setcounter{page}{1}

\tableofcontents
\newpage


%---------------------------------------------------------------------------------
% Inhalt

\pagenumbering{arabic}


\section{Thema}
Auf der Erde existieren 8.616 Gebirgszüge \citep[8]{Snethlage_2022}, etwa 1,2 Millionen benannte Erhebungen
mit einer Höhe von mehr als einem Meter \citep{peakvisor_mountains}, sowie nahezu 1.350 vermutlich aktive
Vulkane \citep{usgs_volcanoes}.
Von diesen sind schätzungsweise 500 in historischen Zeiten nachweislich ausgebrochen \ebd[].
Die Bergbildung und der Vulkanismus auf der Erde sind Gegenstand zahlreicher Studien
und ihr Verständnis gilt als weitgehend etabliert \citep[77-104, 485-506]{Plummer_2015}.

Der italienische Gelehrte, Astronom und Philosoph Galileo Galilei berichtete 1610 erstmals in einer wissenschftlichen
Arbeit über die Existenz von Bergen auf dem Mond und beschrieb damit als Erster Berge, Gebirge und Krater
auf einem anderen Himmelskörper \citep[13]{Galilei_1610}.
Die 1971 zum Mars gestartete Mariner 9 Sonde lieferte erstmals eindeutige Nachweise für Vulkanismus auf dem
erdähnlichen Planeten \citep[]{McCauley_1972}. Zudem wurden die von \citet[]{Peale_1979} formulierten Vorhersagen
zur aktuellen vulkanische Aktivität des Jupitermonds Io durch Beobachtungen der Voyager 1 Raumsonde bestätigt \citep[]{Smith_1979}.
Des Weiteren deuten Schwankungen im Schwefeldioxit Wert der Venus auf aktiven Vulkanismus hin \citep[30]{Nimmo_1998}.
Damit ist die Venus der einzige Planet abseits der Erde mit aktivem Vulkanismus, da ein Solcher auf dem Mars und
dem Merkur als erloschen gilt \parencites[687]{Sigurdsson_2015}[607]{Smrekar_2010}.





%---------------------------------------------------------------------------------
% Verzeichnisse

\newpage
\pagenumbering{Roman}
\setcounter{page}{2}

\nocite{*}                      % gibt alle Quellen aus der .bib an
\printbibliography

\end{document}