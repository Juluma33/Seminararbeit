\documentclass[a4paper,11pt]{article}

\usepackage[ngerman]{babel}
\usepackage{graphicx}
\usepackage{eso-pic}
\usepackage{tabularx}
\usepackage{stackengine}
\usepackage{ragged2e}
\usepackage{url}
\usepackage{hyperref}
\usepackage{microtype}
\usepackage{unicode-math}
\setmathfont{Latin Modern Math}     % gegen Warnung wegen Arial Schriftart (ihr fehlt der Mathepart)
\usepackage[left=3cm, right=5cm, top=1.5cm, bottom=2cm]{geometry}
\usepackage{setspace}
\onehalfspacing                 % Zeilenabstand
\usepackage{fontspec}
\setmainfont{Arial}


\author{}
\date{}
\title{{\fontsize{26pt}{30pt}\selectfont Seminarfach} \\[3cm] {\fontsize{30pt}{40pt}\selectfont Exposé}}


\newcommand{\StartBackground}{
\newgeometry{top=6.5cm,bottom=1.5cm,left=4cm,right=2cm}
\AddToShipoutPictureBG{
\put(0,0){
\parbox[b][\paperheight]{\paperwidth}{
\vfill
\centering
\includegraphics[width=\paperwidth,height=\paperheight]{../Background.png}
\vfill}}}}
\newcommand{\StopBackground}{\newpage\restoregeometry\ClearShipoutPictureBG}

\pagestyle{empty}
\pagenumbering{gobble}


% Zitation
\usepackage{csquotes}

\usepackage[
style=authoryear,
backend=biber,
natbib=true,
maxcitenames=1,
mincitenames=1,
maxbibnames=99,
minbibnames=99,
citetracker=context,
giveninits=true
]{biblatex}

\DeclareDelimFormat{finalnamedelim}{\space und\space}
\renewcommand*{\postnotedelim}{\addcolon\space}
\renewcommand*{\multicitedelim}{\addsemicolon\space}
\DeclareFieldFormat{postnote}{S.#1}
\DefineBibliographyStrings{ngerman}{andothers = {et\addabbrvspace al\adddot}}
\renewbibmacro*{name:andothers}{}
\DeclareFieldFormat[book,thesis]{title}{\mkbibemph{#1}}
\DeclareFieldFormat[article]{title}{\mkbibquote{#1}}
\DeclareNameAlias{author}{family-given}
\DeclareNameAlias{editor}{family-given}
\setlength{\bibhang}{1.25cm}
\setlength{\bibitemsep}{0.5\baselineskip}
\setlength{\bibparsep}{0pt}
\DefineBibliographyStrings{ngerman}{editor = {Hrsg\adddot}, editors = {Hrsg\adddot}}
\DeclareLabeldate{\field{date} \field{year}}
\ExecuteBibliographyOptions{uniquename=false}
\DefineBibliographyStrings{ngerman}{urlseen = {Zugriff am}}
\renewcommand*{\bibfont}{\fontsize{9pt}{11pt}\selectfont}      % Größe der Verzeichnisses
\newcommand{\ebd}[1][]{\textit{(ebd.\if\relax\detokenize{#1}\relax\else~S.#1\fi)}}  % ebd. Zitation (geht leider nicht automatisch)
\AtBeginBibliography{\interlinepenalty=10000}
\addbibresource{../Quellen/Quellen.bib}                 % --> Ändern wenn in Unterordnern


%---------------------------------------------------------------------------------
% Titelseite / Benotung

\begin{document}
\justifying                 % => für Blocksatz
\StartBackground

\maketitle
\vspace{5cm}
\begin{table}[h!]
    \renewcommand{\arraystretch}{3}
    \begin{tabularx}{\textwidth}{>{\raggedright\arraybackslash}p{5cm}
                                 >{\raggedright\arraybackslash}X}
        \textbf{Thema:}&Entstehungsmechanismen und strukturelle Merkmale von Zentralbergen in Impaktkratern, Vulkanen und Äquatorialkämmen im Sonnensystem\\
        \textbf{Verfasser:}& Julian Luke Mawill Baden \\
        \textbf{Fachlehrkraft und Kursbezeichnung:}& ROP \quad / \quad SF2\\
        \end{tabularx}
\end{table}


\newpage
\noindent\textbf{Gymnasium Mellendorf\\Fritz-Sennheiser-Platz 2\\30900 Wedemark}\\
Schuljahr:  2025/26\\

{\fontsize{26pt}{20pt}\selectfont\begin{center} Seminarfach \end{center}}

\noindent\rule{\linewidth}{0.5pt}

\begin{table}[h!]
    \renewcommand{\arraystretch}{2}
    \begin{tabularx}{\textwidth}{>{\raggedright\arraybackslash}p{4.5cm}
                                 >{\raggedright\arraybackslash}X}
        Verfasser:& Julian Luke Mawill Baden \\
        Thema:&Entstehungsmechanismen und strukturelle Merkmale von Zentralbergen in Impaktkratern, Vulkanen und Äquatorialkämmen im Sonnensystem\\
        Fachlehrkraft und Kursbezeichnung:& ROP \quad / \quad SF2\\
        \end{tabularx}
\end{table}

\noindent\rule{\linewidth}{0.5pt}\\[0.5cm]
\textbf{Abgabetermin: 19.02.2026 zu Beginn der Seminarfachsitzung um 14 Uhr}\\[0.3cm]
\rule{\linewidth}{0.5pt}\\

\begin{tabular}{l  l}
    &\\
    \textbf{Benotung:}&\fbox{\rule[-1.5em]{0pt}{4em}\hspace{3cm}}\\
    &\\[0.3cm]
\end{tabular}

\hspace{3cm} Verfasser*in\\[1cm]

\hspace{1cm} \stackunder[4pt]{\rule{4.5cm}{0.6pt}}{\makebox[4.5cm]{Datum}}
\hspace{2cm} \stackunder[4pt]{\rule{5.5cm}{0.6pt}}{\makebox[5.5cm]{Unterschrift der Lehrkraft}}


%---------------------------------------------------------------------------------
% Eigenständigkeit --> HINTEN ANHÄNGEN!!!

\newpage
\noindent{\fontsize{14pt}{16pt}\selectfont \textbf{Erklärung des Verfassers}}\\[0.5cm]
Hiermit erkläre ich, dass ich die vorliegende Arbeit selbstständig angefertigt,
keine anderen als die angegebenen Hilfsmittel benutzt und die Stellen der Facharbeit,
die im Wortlaut oder im wesentlichen Inhalt aus anderen Werken oder dem Internet entnommen wurden,
mit genauer Quellenangabe kenntlich gemacht habe.\\[0.8cm]

\noindent Verfasser:\quad Julian Luke Mawill Baden\\[0.8cm]

\hspace{1cm} \stackunder[4pt]{\rule{4.5cm}{0.6pt}}{\makebox[4.5cm]{Datum}}
\hspace{2cm} \stackunder[4pt]{\rule{5.5cm}{0.6pt}}{\makebox[5.5cm]{Unterschrift}}

\StopBackground


%---------------------------------------------------------------------------------
% Inhaltsverzeichnis


\normalsize

\pagestyle{plain}
\pagenumbering{Roman}
\setcounter{page}{1}

\tableofcontents
\newpage


%---------------------------------------------------------------------------------
% Inhalt

\onehalfspacing
\pagenumbering{arabic}


\section{Thema}

% kleine Einführung als erste Begründung für das Thema
Die terrestrische Topographie ist durch eine große Vielfalt gekennzeichnet:
so existieren auf der Erde 8.616 Gebirgszüge \citep[8]{Snethlage_2022}, sowie eine Menge von etwa 1,2 Millionen benannte
Erhebungen mit einer vertikalen Ausdehnung von mehr als einem Meter \citep{peakvisor_mountains}.
Neben diesen tektonischen Strukturen lassen sich potentiell 1.350 vermutlich aktive Vulkane identifizieren \citep{usgs_volcanoes}.
Von diesen weisen historische Aufzeichnungen für circa 500 Vulkane Ausbrüche aus \ebd[].

Die Prozesse der Orogenese (Bergbildung) und die Mechanismen des terrestrischen Vulkanismus auf der Erde sind allerdings bereits
seit langem Gegenstand intensiver Forschungen \citep[77-104, S.485-506]{Plummer_2015}. Auf Grundlage umfangreicher geophysikalischer
und geochemischer Untersuchungen gilt daher heute das grundlegende Verständnis der zugrunde liegender Prozesse als weitgehend
etabliert \ebd[].

\subsection{Zentralberge}
Die Ausweitung dieser geologischen Betrachtung auf extraterrestrische Körper stellt einen Meilenstein in der Astronomie dar.
Im Jahr 1610 publizierte der italienische Gelehrte, Astronom und Philosoph Galileo Galilei Beobachtungen, in denen er
erstmals in einer wissenschftlichen Arbeit die Existenz von Bergen auf dem Mond dokumentierte \citep[13]{Galilei_1610}.
Damit beschrieb er als Erster Berge, Gebirge und Krater auf einem anderen Himmelskörper \ebd[].

Gegenwärtig besteht ein deutlich detailierteres Verständnis der Prozesse der Kraterbildung. Diese werden typischerweise in drei
Phasen unterteilt: die Kontakt- und Kompressionsphase, die Aushöhlungsphase (Excavation phase) sowie die Modifikationsphase
\citep[6-11]{Rae_2018}. Das Ergebnis der ersten beiden Phasen ist eine transiente konkave Aushöhlung. Aufgrund gravitativer
Kräfte kollabiert die Struktur teilweise, was zur endgültigen Kratermorphologie führt und als Modifikationsphase bezeichnet
wird \ebd[10]. Wärend dieser Phase kann aufgrund von zurückfederndem Material in mittelgroßen, komplexen Kratern einen Zentralberg
in der Kratermitte bilden \ebd[12]. Bei sehr großen Impaktstrukturen kann diese Erhebung allerdings instabil werden,
wodurch nach dem Kollaps anstatt eines Zentralberges ein ringförmiges Relief, ein sogenannter Zentralring, zurückbleibt \ebd[156].
Ein prominentes Beispiel hierfür ist der Chicxulub-Krater auf der Yucatán Halbinsel \ebd[].


\subsection{Vulkanismus}
In der jüngeren Geschichte der Raumfahrt erweiterte sich das Wissen über den extraterrestrischen Vulkanismus stark.
Die 1971 zum Mars gestartete Mariner 9 Sonde lieferte erstmals zweifelsfreie Belege für vulkanische Aktivitäten auf dem
roten Planeten \citep[292]{McCauley_1972}. In diesem Zusammenhang wurden primär vier Schildvulkane innehalb der
Tharsis-Amazonis-Elysium-Region charakterisiert, wobei gerade die morphologischen Ähnlichkeit zu terrestrische Vulkanen
genannt werden \ebd[].

Parallel dazu konnten die theoretischen Vorhersagen von \citet[894]{Peale_1979} bezüglich einer
rezenten vulkanische Aktivität des Jupitermonds Io durch Bilddaten der Voyager 1 Raumsonde bestätigt werden \citep[961]{Smith_1979}.
diese ausergewöhnlich hohe Aktivität besteht vermutlich schon seit der Frühzeit des Sonnensystems vor etwa 4,57 Milliarden Jahren
\citep[1]{Kleer_2024} $[!!!Noch Besorgen!!!]$. Infolge dessen gilt der Jupitermond heute als der vulkanisch aktivste
bekannte Himmelskörper des Sonnensystem \citep[]{NASA_2026_Io}.

Ein weiteres Forschungsfeld stellt die Venus dar, in deren Atmosphäre Fluktuationen in der Schwefeldioxit-Konzentration $(SO_2)$
auf gegenwärtigen Vulkanismus hindeuten \citep[30]{Nimmo_1998}. Diese Indizien wurden durch die Analysen von
\citet[607]{Smrekar_2010} untermauert, indem Lavaströme auf ein geologisches Alter von einigen hundert bis zehntausend Jahren
datiert werden konnten. Dadurch gilt die Venus als der einzige planetare Körper neben der Erde, der gegenwärtig aktive vulkanische
Prozesse aufweist, während entsprechende Phänomene auf dem Mars und dem Merkur als erloschen gelten \citep[687]{Sigurdsson_2015}.

\subsection{Äquatorialkämme}
Zusätzlich zu Vulkanen und Zentralbergen in Impaktkratern weist der Saturnmond Iapetus ein im Sonnensystem einzigartiges Gebirge
auf. 
$[Länge des Kamms]$




\section{Fragestellung}





\section{Ziel der Arbeit}




\section{Gliederung und Methodik}
Die geplante Seminararbeit wird thematisch, analog zum Themenabschnitt dieses Exposés, in drei Hauptkathgorien aufeteilt sein:
Zentralbergen in Impaktkratern, Vulkanismus und Äquatorialkämme. Diese Kategorien werden jeweils hinsichtlich ihrer
Entstehungsprozesse, morphologischen und geologischen Merkmale systematisch untersucht. Des Weiteren soll im zweiten Teil der Arbeit
eine Analyse vergleichend Gemeinsamkeiten und Unterschieden dieser Kategorien, sowie zwischen Vertretern innerhalb der Kategorien
herausarbeiten.

Methodisch basiert die Arbeit auf Literaturrecherche und -auswertung. Dabei werden sowohl grundlegende Werke zur Bergbildung
und Vulkanforschung auf anderen Himmelskörpern, als auch aktuelle Fachartikel berücksichtigt.

Für jede dieser Kategorien sollen zunächst die theoretischen Grundlagen der Bildung auf Basis aktueller und älterer Publikationen
besprochen werden. Anschließend werden die charakteristischen geometrischen, physikalischen und geologischen Eigenschaften
beschrieben. Dafür sollen sowohl Daten aus Fernerkundung (beispielsweise Sondenbilder und topographische Karten),
sowie moderne Modelle zur Krater-, Vulkan- und Äquatorialkammbildung verwendet werden,
um ein umfassendes Bild über diese Prozesse zu liefern.
Zur besseren Darstellung dieser Modelle und Daten werden Grafiken, Darstellungen und Diagramme verwendet, um eine Bessere
vorstellung der beschriebenen Strukturen zu ermöglichen.

Im zweiten Teil der Arbeit soll eine vergleichende Analyse Unterschieden und Parallelen zwischen den Kategorien, sowie
innerhalb ihrer Vertreter herausarbeiten, um Rückschlüsse auf ihre Entstehung und Entwicklung zu ziehen.





\section{Zeitplan}






%---------------------------------------------------------------------------------
% Verzeichnisse

\newpage
\pagenumbering{Roman}
\setcounter{page}{2}

\nocite{*}                      % gibt alle Quellen aus der .bib an (gewollt??)
\printbibliography[title={Quellenverzeichnis}]

\end{document}