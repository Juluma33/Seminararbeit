\documentclass[a4paper,11pt]{article}

\usepackage[ngerman]{babel}
\usepackage{graphicx}
\usepackage{eso-pic}
\usepackage{tabularx}
\usepackage{stackengine}

\usepackage[left=3cm, right=5cm, top=1.5cm, bottom=2cm]{geometry}

\usepackage{fontspec}
\setmainfont{Arial}

\author{}
\date{}
\title{{\fontsize{26pt}{30pt}\selectfont Seminarfach} \\[3cm] {\fontsize{30pt}{40pt}\selectfont Literaturliste für das Exposé}}


\newcommand{\StartBackground}{
\newgeometry{top=6.5cm,bottom=1.5cm,left=4cm,right=2cm}
\AddToShipoutPictureBG{
\put(0,0){
\parbox[b][\paperheight]{\paperwidth}{
\vfill
\centering
\includegraphics[width=\paperwidth,height=\paperheight]{../Background.png}
\vfill}}}}
\newcommand{\StopBackground}{\newpage\restoregeometry\ClearShipoutPictureBG}

\pagestyle{empty}
\pagenumbering{gobble}


% Zitation
\usepackage{csquotes}

\usepackage[
style=authoryear,
backend=biber,
natbib=true,
maxcitenames=2,
mincitenames=1,
citetracker=context,
giveninits=true
]{biblatex}

\DeclareDelimFormat{finalnamedelim}{\space und\space}
\renewcommand*{\postnotedelim}{\addcomma\space}
\DeclareFieldFormat{postnote}{S.#1}
\renewcommand*{\multicitedelim}{\addsemicolon\space}
\DefineBibliographyStrings{ngerman}{andothers = {et\addabbrvspace al\adddot}}
\ExecuteBibliographyOptions{doi=true, url=false}
\DeclareFieldFormat[article,book,thesis]{title}{\mkbibemph{#1}}


\addbibresource{../Quellen/Quellen.bib}                 % --> Ändern wenn in Unterordnern


%---------------------------------------------------------------------------------
% Titelseite / Benotung

\begin{document}
\StartBackground

\maketitle
\vspace{4cm}
\begin{table}[h!]
    \renewcommand{\arraystretch}{3}
    \begin{tabularx}{\textwidth}{>{\centering\arraybackslash}l
                                 >{\raggedright\arraybackslash}X}
        \textbf{Thema:}&Entstehungsmechanismen und strukturelle Merkmale von Zentralbergen in Impaktkratern, Vulkanen und Äquatorialkämmen im Sonnensystem\\
        \textbf{Verfasser:}&\\
        \textbf{Fachlehrkraft:}& \quad ()
        \end{tabularx}
\end{table}


\newpage
\noindent\textbf{Gymnasium Mellendorf\\Fritz-Sennheiser-Platz 2\\30900 Wedemark}\\
Schuljahr:  2025/26\\

{\fontsize{26pt}{20pt}\selectfont\begin{center} Seminarfach \end{center}}

\noindent\rule{\linewidth}{0.5pt}

\begin{table}[h!]
    \renewcommand{\arraystretch}{2}
    \begin{tabularx}{\textwidth}{>{\centering\arraybackslash}l
                                 >{\raggedright\arraybackslash}X}
        Verfasser:&\\
        Thema:&Entstehungsmechanismen und strukturelle Merkmale von Zentralbergen in Impaktkratern, Vulkanen und Äquatorialkämmen im Sonnensystem\\
        Fachlehrkraft:& \quad ()
        \end{tabularx}
\end{table}

\noindent\rule{\linewidth}{0.5pt}\\[0.5cm]
\textbf{Abgabetermin: 29.01.2026}\\[0.3cm]
\rule{\linewidth}{0.5pt}\\

\begin{tabular}{l  l}
    &\\
    \textbf{Benotung:}&\fbox{\rule[-1.5em]{0pt}{4em}\hspace{3cm}}\\
    &\\[0.3cm]
\end{tabular}

\hspace{3cm} Verfasser*in\\[1cm]

\hspace{1cm} \stackunder[4pt]{\rule{4.5cm}{0.6pt}}{\makebox[4.5cm]{Datum}}
\hspace{2cm} \stackunder[4pt]{\rule{5.5cm}{0.6pt}}{\makebox[5.5cm]{Unterschrift der Lehrkraft}}


\StopBackground


%---------------------------------------------------------------------------------
% Inhalt


\fontsize{11pt}{16.5pt}
\pagestyle{plain}
\pagenumbering{arabic}
\setcounter{page}{1}



\begin{table}[h!]
    \renewcommand{\arraystretch}{2}
    \begin{tabularx}{\textwidth}{>{\centering\arraybackslash}l
                                 >{\raggedright\arraybackslash}X}
        %--------------------------------------------------------------------------------------------------------------------
        \textbf{Zentralberge in Impaktkratern:}\\
        \citeauthor{Rae_2018}, \citeyear{Rae_2018}&\\
        
        %--------------------------------------------------------------------------------------------------------------------
        \textbf{Vulkane:}\\
        
        %--------------------------------------------------------------------------------------------------------------------
        \textbf{Äquatorialkämmen:}\\
        
        \end{tabularx}
\end{table}





%---------------------------------------------------------------------------------
% Verzeichnisse

\newpage
\pagenumbering{Roman}
\setcounter{page}{2}
\printbibliography

\end{document}