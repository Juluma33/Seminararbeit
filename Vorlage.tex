\documentclass[a4paper,11pt]{article}

\usepackage[ngerman]{babel}
\usepackage{graphicx}
\usepackage{eso-pic}
\usepackage{graphicx}
\usepackage{setspace}
\usepackage{fancyhdr}
\usepackage{tabularx}

\usepackage[left=3cm, right=5cm, top=1.5cm, bottom=1.5cm]{geometry}
\onehalfspacing

\usepackage{fontspec}
\setmainfont{Arial}

\author{}
\date{}
\title{{\fontsize{26pt}{30pt}\selectfont Seminarfach} \\[3cm] {\fontsize{30pt}{40}\selectfont Exposé}}


% Commands for the pages with background
\newcommand{\BackgroundImageFile}{Background.png}

\newcommand{\StartBackground}{
\newgeometry{top=5cm,bottom=1.5cm,left=4cm,right=2cm}
\AddToShipoutPictureBG*{
\put(0,0){
\parbox[b][\paperheight]{\paperwidth}{
\vfill
\centering
\includegraphics[width=\paperwidth,height=\paperheight]{\BackgroundImageFile}%
\vfill}}}}

\newcommand{\StopBackground}{\newpage\restoregeometry\ClearShipoutPictureBG}


\pagestyle{empty}
\pagenumbering{gobble}



%---------------------------------------------------------------------------------
% Titelseite

\begin{document}
\StartBackground

\maketitle
\vspace{6cm}
\begin{table}[h!]
    \renewcommand{\arraystretch}{3}
    \begin{tabularx}{\textwidth}{>{\centering\arraybackslash}l
                                 >{\raggedright\arraybackslash}X}
        \textbf{Thema:}&Entstehungsmechanismen und strukturelle Merkmale von Zentralbergen in Impaktkratern, Vulkanen und Äquatorialkämmen im Sonnensystem\\
        \textbf{Verfasser:}&\\
        \textbf{Fachlehrkraft:}& (ROP)
        \end{tabularx}
    \label{Grade der Knoten}
\end{table}

\StopBackground

%---------------------------------------------------------------------------------


Inhalt





\end{document}