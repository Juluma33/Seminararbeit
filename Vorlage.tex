\documentclass[a4paper,11pt]{article}

\usepackage[ngerman]{babel}
\usepackage{graphicx}
\usepackage{eso-pic}
\usepackage{setspace}
\usepackage{tabularx}
\usepackage{stackengine}

\usepackage[left=3cm, right=5cm, top=1.5cm, bottom=1.5cm]{geometry}
\onehalfspacing

\usepackage{fontspec}
\setmainfont{Arial}

\author{}
\date{}
\title{{\fontsize{26pt}{30pt}\selectfont Seminarfach} \\[3cm] {\fontsize{30pt}{40pt}\selectfont Exposé}}


% Commands for the pages with background
\newcommand{\BackgroundImageFile}{Background.png}

\newcommand{\StartBackground}{
\newgeometry{top=6.5cm,bottom=1.5cm,left=4cm,right=2cm}
\AddToShipoutPictureBG{
\put(0,0){
\parbox[b][\paperheight]{\paperwidth}{
\vfill
\centering
\includegraphics[width=\paperwidth,height=\paperheight]{\BackgroundImageFile}%
\vfill}}}}

\newcommand{\StopBackground}{\newpage\restoregeometry\ClearShipoutPictureBG}


\pagestyle{empty}
\pagenumbering{gobble}



%---------------------------------------------------------------------------------
% Titelseite / Benotung

\begin{document}
\StartBackground

\maketitle
\vspace{4cm}
\begin{table}[h!]
    \renewcommand{\arraystretch}{3}
    \begin{tabularx}{\textwidth}{>{\centering\arraybackslash}l
                                 >{\raggedright\arraybackslash}X}
        \textbf{Thema:}&Entstehungsmechanismen und strukturelle Merkmale von Zentralbergen in Impaktkratern, Vulkanen und Äquatorialkämmen im Sonnensystem\\
        \textbf{Verfasser:}&Placeholder\\
        \textbf{Fachlehrkraft:}&Placeholder \quad (Placeholder)
        \end{tabularx}
\end{table}


\newpage
\noindent\textbf{Gymnasium Mellendorf\\Fritz-Sennheiser-Platz 2\\30900 Wedemark}\\
Schuljahr:  2025/26\\

{\fontsize{26pt}{20pt}\selectfont\begin{center} Seminarfach \end{center}}

\noindent\rule{\linewidth}{0.5pt}

\begin{table}[h!]
    \renewcommand{\arraystretch}{2}
    \begin{tabularx}{\textwidth}{>{\centering\arraybackslash}l
                                 >{\raggedright\arraybackslash}X}
        Verfasser:&Placeholder\\
        Thema:&Entstehungsmechanismen und strukturelle Merkmale von Zentralbergen in Impaktkratern, Vulkanen und Äquatorialkämmen im Sonnensystem\\
        Fachlehrkraft:& Placeholder \quad (Placeholder)
        \end{tabularx}
\end{table}

\noindent\rule{\linewidth}{0.5pt}\\[0.5cm]
\textbf{Abgabetermin: 19.02.2026 zu Beginn der Seminarfachsitzung um 14 Uhr}\\[0.3cm]
\rule{\linewidth}{0.5pt}\\

\begin{tabular}{l  l}
    &\\
    \textbf{Benotung:}&\fbox{\rule[-1.5em]{0pt}{4em}\hspace{3cm}}\\
    &\\[0.3cm]
\end{tabular}

\hspace{3cm} Verfasser*in\\[1cm]

\hspace{1cm} \stackunder[4pt]{\rule{4.5cm}{0.6pt}}{\makebox[4.5cm]{Datum}}
\hspace{2cm} \stackunder[4pt]{\rule{5.5cm}{0.6pt}}{\makebox[5.5cm]{Unterschrift der Lehrkraft}}


%---------------------------------------------------------------------------------
% Eigenständigkeit --> HINTEN ANHÄNGEN!!!

\newpage
\noindent{\fontsize{14pt}{16pt}\selectfont \textbf{Erklärung des Verfassers}}\\[0.5cm]
Hiermit erkläre ich, dass ich die vorliegende Arbeit selbstständig angefertigt,
keine anderen als die angegebenen Hilfsmittel benutzt und die Stellen der Facharbeit,
die im Wortlaut oder im wesentlichen Inhalt aus anderen Werken oder dem Internet entnommen wurden,
mit genauer Quellenangabe kenntlich gemacht habe.\\[0.8cm]

\noindent Verfasser:\quad Placeholder\\[0.8cm]

\hspace{1cm} \stackunder[4pt]{\rule{4.5cm}{0.6pt}}{\makebox[4.5cm]{Datum}}
\hspace{2cm} \stackunder[4pt]{\rule{5.5cm}{0.6pt}}{\makebox[5.5cm]{Unterschrift}}


\StopBackground


%---------------------------------------------------------------------------------


Inhalt





\end{document}